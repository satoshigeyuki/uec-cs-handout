\documentclass[dvipdfmx]{cs-handout}

%Font Info
\usepackage[T1]{fontenc}
\usepackage{otf}
\usepackage{color}
\usepackage[dvips]{graphicx}
\usepackage{latexsym}
\usepackage{listings,jvlisting}
\usepackage{url}
\usepackage{here}
\usepackage{paralist}

\newcommand{\Note}[1]{\noindent \textbf{\textcolor{blue}{#1}}}

%Document Info
\title{柔軟で動的な実行環境を提供するための\\コンテナ型仮想化基盤アーキテクチャの検討}
\author{平地浩一}
\stdnum{1234567}
\office{矢崎研究室}
%\lhead{20YY年度 情報理工学研究科 情報・ネットワーク工学専攻 CSプログラム 修士論文(中間)発表}
\lhead{2024年度 情報理工学域 I類 CSプログラム 卒業研究(中間)発表}
\rhead{2024/10/3}

\begin{document}
\maketitle

\section{はじめに}
%\Note{研究ストーリー(このフォントはコメント)}

近年,仮想化技術の発展により,物理的な計算資源を仮想化し,1台の物理マシンで複数の仮想マシンを実行することができるようになった.
仮想化される主な資源としてCPU,メモリ,ネットワーク,ストレージなどがある.
仮想化技術の中でも,コンテナ仮想化は従来の仮想マシンと比べてオーバーヘッドが小さく,特にシステム開発や運用においては,柔軟性,可搬性,再現性に優れるなど多くの利点がある.
また,物理的なサーバリソースを効率的に運用したり,拡張性やセキュリティに優れたシステムを構築することも可能であり,今後より多くの分野での活用が期待される.

最近では複数のコンテナを管理・運用する統合管理(オーケストレーション)の代表的な仕組みとして Kubernetes (k8s) が広く活用されている.
k8s は Google が開発したコンテナの管理・運用基盤である.
現在は Cloud Native Foundation のプロジェクトとして開発されているオープンソースのシステムである\cite{k8s}.
k8s を活用することでスケーラブルでシステム障害に強いコンテナベースのシステムを構築することが可能であり,多くのシステムで活用されている.
k8s はシステム運用だけでなくネットワークセキュリティや HPC といった分野でも活用が期待されている.
Eriksson らは k8s を用いてハニーポッドを構築することでクラウドシステムに対して脅威インテリジェンスとセキュリティ意識を提供する試みをしている\cite{Eriksson2023}.
Fultonらは完全にコンテナ化された HPC システムを検討している\cite{Fulton2023}.

一方で,k8s を初めとするコンテナ型仮想化技術およびその管理機構は複雑化しており,多くの開発者や研究者にとって活用のハードルが高い.
このことから,広く様々な分野でコンテナ方仮想化技術を応用するためには,利用者の視点からより手軽に扱えるコンテナ型仮想化基盤の開発が求められている.

\section{準備}
\Note{前提知識の説明}

ああああああああああああああああああああああああああああああああああああ
ああああああああああああああああああああああああああああああああああああ
ああああああああああああああああああああああああああああああああああああ
ああああああああああああああああああああああああああああああああああああ
ああああああああああああああああああああああああああああああああああああ
ああああああああああああああああああああああああああああああああああああ
ああああああああああああああああああああああああああああああああああああ
ああああああああああああああああああああああああああああああああああああ
ああああああああああああああああああああああああああああああああああああ
ああああああああああああああああああああああああああああああああああああ
ああああああああああああああああああああああああああああああああああああ
ああああ

\section{提案}
\Note{提案内容の説明}

ああああああああああああああああああああああああああああああああああああ
ああああああああああああああああああああああああああああああああああああ
ああああああああああああああああああああああああああああああああああああ
ああああああああああああああああああああああああああああああああああああ
ああああああああああああああああああああああああああああああああああああ
ああああああああああああああああああああああああああああああああああああ
ああああああああああああああああああああああああああああああああああああ
ああああああああああああああああああああああああああああああああああああ
ああああああああああああああああああああああああああああああああああああ
ああああああああああああああああああああああああああああああああああああ
ああああああああああああああああああああああああああああああああああああ
ああああ

ああああああああああああああああああああああああああああああああああああ
ああああああああああああああああああああああああああああああああああああ
ああああああああああああああああああああああああああああああああああああ
ああああああああああああああああああああああああああああああああああああ
ああああああああああああああああああああああああああああああああああああ
ああああああああああああああああああああああああああああああああああああ
ああああああああああああああああああああああああああああああああああああ
ああああああああああああああああああああああああああああああああああああ
ああああああああああああああああああああああああああああああああああああ
ああああああああああああああああああああああああああああああああああああ
ああああああああああああああああああああああああああああああああああああ
ああああ

\section{評価}
\Note{提案に対する評価}

ああああああああああああああああああああああああああああああああああああ
ああああああああああああああああああああああああああああああああああああ
ああああああああああああああああああああああああああああああああああああ
ああああああああああああああああああああああああああああああああああああ
ああああああああああああああああああああああああああああああああああああ
ああああああああああああああああああああああああああああああああああああ
ああああああああああああああああああああああああああああああああああああ
ああああああああああああああああああああああああああああああああああああ
ああああああああああああああああああああああああああああああああああああ
ああああああああああああああああああああああああああああああああああああ
ああああああああああああああああああああああああああああああああああああ
ああああ

ああああああああああああああああああああああああああああああああああああ
ああああああああああああああああああああああああああああああああああああ
ああああああああああああああああああああああああああああああああああああ
ああああああああああああああああああああああああああああああああああああ
ああああああああああああああああああああああああああああああああああああ
ああああああああああああああああああああああああああああああああああああ
ああああああああああああああああああああああああああああああああああああ
ああああああああああああああああああああああああああああああああああああ
ああああああああああああああああああああああああああああああああああああ
ああああああああああああああああああああああああああああああああああああ
ああああああああああああああああああああああああああああああああああああ
ああああ

\section{関連研究}
\Note{参考文献はすべて本文中で引用すべし}

ああああああああああああああああああああああああああああああああああああ
ああああああああああああああああああああああああああああああああああああ
ああああああああああああああああああああああああああああああああああああ
ああああああああああああああああああああああああああああああああああああ
ああああああああああああああああああああああああああああああああああああ
ああああああああああああああああああああああああああああああああああああ
ああああああああああああああああああああああああああああああああああああ
ああああああああああああああああああああああああああああああああああああ
ああああああああああああああああああああああああああああああああああああ
ああああああああああああああああああああああああああああああああああああ
ああああああああああああああああああああああああああああああああああああ
ああああ

\section{おわりに}
\Note{まとめと今後の課題}

ああああああああああああああああああああああああああああああああああああ
ああああああああああああああああああああああああああああああああああああ
ああああああああああああああああああああああああああああああああああああ
ああああああああああああああああああああああああああああああああああああ
ああああああああああああああああああああああああああああああああああああ
ああああああああああああああああああああああああああああああああああああ
ああああああああああああああああああああああああああああああああああああ
ああああああああああああああああああああああああああああああああああああ
ああああああああああああああああああああああああああああああああああああ
ああああああああああああああああああああああああああああああああああああ
ああああああああああああああああああああああああああああああああああああ
ああああ

%\bibliographystyle{ipsjunsrt}
\bibliographystyle{unsrt}
\bibliography{export}

\end{document}