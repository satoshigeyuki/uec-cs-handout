\documentclass[dvipdfmx,twocolumn]{jsarticle}
\usepackage{cs-handout}

% フォント関連
\usepackage[T1]{fontenc}
\usepackage{otf}

% 便利なパッケージ
\usepackage{graphicx}
\usepackage{xcolor}
\usepackage{booktabs}
\usepackage{url}
\usepackage{hyperref}
\usepackage{pxjahyper}

\usepackage{pgffor}
\newcommand{\RepeatPlaceholder}[2]{\textcolor{gray}{\foreach \x in {1,...,#1}{#2}}}
\newcommand{\Note}[1]{\noindent \textbf{\textcolor{blue}{#1}}}

% 著者情報
\title{研究タイトル}
\author{椎江洲太郎}
\studentid{1234567}
\office{〇〇〇〇研究室} % 〇〇〇〇は主任指導教員名

% 発表会情報
\lhead{\arabic{academicyear}年度 情報理工学研究科 情報・ネットワーク工学専攻 CSプログラム 修士論文発表}
%\lhead{\arabic{academicyear}年度 情報理工学研究科 情報・ネットワーク工学専攻 CSプログラム 修士論文中間発表}
%\lhead{\arabic{academicyear}年度 情報理工学域 I類 CSプログラム 卒業研究発表}
%\lhead{\arabic{academicyear}年度 情報理工学域 I類 CSプログラム 卒業研究中間発表}
\presentationdate{2025}{2}{4} % 発表会の日付

\begin{document}
\maketitle

\section{はじめに}
\Note{研究ストーリー(このフォントはコメント)}

\RepeatPlaceholder{200}{あ}

\RepeatPlaceholder{200}{あ}

\RepeatPlaceholder{200}{あ}

\section{準備}
\Note{前提知識の説明}

\RepeatPlaceholder{200}{あ}

\RepeatPlaceholder{200}{あ}

\RepeatPlaceholder{200}{あ}

\section{提案}
\Note{提案内容の説明}

\begin{figure}[tb]
 \centering
 \includegraphics[width=\linewidth]{example-image-a}
 \caption{\Note{図のキャプションは下}}
\end{figure}

\RepeatPlaceholder{200}{あ}

\RepeatPlaceholder{200}{あ}

\RepeatPlaceholder{200}{あ}

\section{評価}
\Note{提案に対する評価}

\begin{table}[tb]
 \centering
 \caption{\Note{表のキャプションは上}}
 \vspace{-\bigskipamount}
 \begin{tabular}[t]{ll}
  \toprule
  パッケージ名 & 説明 \\ \midrule
  graphicx & 画像を挿入する \\
  xcolor & 色を指定する \\
  booktabs & このスタイルの表を記述する \\
  url & URL \url{uec.ac.jp} を記述する \\
  hyperref & URL をハイパーリンクにする \\
  pxjahyper & hyperrefの日本語向け調整 \\
  \bottomrule
 \end{tabular}
\end{table}

\RepeatPlaceholder{200}{あ}

\RepeatPlaceholder{200}{あ}

\RepeatPlaceholder{200}{あ}

\section{関連研究}
\Note{参考文献はすべて本文中で適切に引用すべし:
例えば,角田~\cite{角田24},小宮ら~\cite{小宮24},J17-CS~\cite{ipsj-csec18:J17-CS},CS2023~\cite{kumar24:CS2023}.}

\RepeatPlaceholder{200}{あ}

\RepeatPlaceholder{200}{あ}

\section{おわりに}
\Note{まとめと今後の課題}

\RepeatPlaceholder{200}{あ}


\bibliographystyle{jplain}
\bibliography{ref}
\Note{pBibTeXの利用を推奨}

\end{document}
